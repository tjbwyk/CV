\documentclass{article}
\usepackage{amsmath}
\usepackage{graphicx}

\begin{document}

\section{Assignment 1 -- ICP Algorithm}
\label{sec:assignment1}

This section describes the implementation of ICP algorithm and its variants in order to build 3D model from given set of point clouds. In addition, we also analyze 2 different point merging strategies and compare the results. The performance of ICP is measured using Root Mean Square (RMS), as defined

\begin{align} 
\begin{split}
RMS(A_{1},A_{2},\psi) = \sqrt{\frac{1}{n}\sum_{a \in A_{1}}^{} \left \| a - \psi (a) \right \|^{2}}
\end{split}					
\end{align}

The following sections are organized as follows. Section~\ref{subsec:basic_icp} describes implementation of basic ICP algorithm and its variants. We study to compare between two different point selection techniques and analyze the results. Section~\ref{subsec:merging_scenes} explains our experiments of two merging strategies. In the first experiment, we estimate camera poses using each two consecutive frames. Then, by using those estimated camera poses we merge all the point clouds into one point cloud and visualize the result. In the second experiment, we iteratively estimate camera pose for consecutive frames and merge the point clouds simultaneously. At the end of iteration, we will get a 3D model as the result.

\subsection{Implementing ICP Algorithm}
\label{subsec:basic_icp}

Lorem ipsum dolor sit amet, consectetuer adipiscing elit. 

\subsubsection{ICP variants}

Lorem ipsum dolor sit amet, consectetuer adipiscing elit. 

\subsection{Merging Scenes}
\label{subsec:merging_scenes}

Lorem ipsum dolor sit amet, consectetuer adipiscing elit. 

\end{document}